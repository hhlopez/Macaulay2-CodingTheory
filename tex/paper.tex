\documentclass[12pt]{amsart}
\frenchspacing \mathsurround=1pt \emergencystretch=5pt
\tolerance=400

\topmargin = -0.1in \evensidemargin = 0.2in \oddsidemargin = 0.2in
\textheight = 23cm \headheight = 10pt
\textwidth = 15.7cm

\usepackage{amsmath, amsthm, latexsym, amssymb, amsfonts, epsfig, epsf, xcolor, hyperref}
\usepackage{mathtools}
\usepackage{bm,xcolor,tikz,hyperref}
%\usepackage{refcheck}
\usepackage{bm}
\usepackage{blkarray}
\usepackage{booktabs}
\usepackage{enumitem}


\newenvironment{pf}{\proof[\proofname]}{\endproof}
\theoremstyle{plain}
\newtheorem{theorem}{Theorem}[section]
\newtheorem{lemma}[theorem]{Lemma}
\newtheorem{corollary}[theorem]{Corollary}
\newtheorem{definition}[theorem]{Definition}
\newtheorem{proposition}[theorem]{Proposition}
\newtheorem{example}[theorem]{Example}
\newtheorem{remark}[theorem]{Remark}

\newcommand{\rmv}[1]{}


\begin{document}
\title{Coding theory package for {\it Macaulay2}}
%%%%%%%%%%%%%%%%%%%%%%%
\author{Taylor Ball}
\address[Taylor Ball]{Department of Mathematics\\University of Notre Dame\\ Notre Dame, IN USA}
\email{trball13@gmail.com}
%%%%%%%%%%%%%%%%%%%%%%%
\author{Eduardo Camps}
\address[Eduardo Camps]{Escuela Superior de F\'isica y Matem\'aticas \\ Instituto Polit\'ecnico Nacional\\ Mexico City, Mexico}
\email{camps@esfm.ipn.mx}
%%%%%%%%%%%%%%%%%%%%%%%
\author{Henry Chimal-Dzul}
\address[Henry Chimal-Dzul]{Department of Mathematics and Center of Ring Theory and its Applications\\ Ohio University\\ Athens, OH USA}
\email{hc118813@ohio.edu}
%%%%%%%%%%%%%%%%%%%%%%%
\author{Delio Jaramillo-Velez}
\address[Delio Jaramillo-Velez]{
Departamento de
Matem\'aticas\\
Centro de Investigaci\'on y de Estudios
Avanzados del
IPN\\
Apartado Postal
14--740 \\
07000 Mexico City, D.F.
}

\email{djaramillo@math.cinvestav.mx}
%%%%%%%%%%%%%%%%%%%%%%%
\author{Hiram H. L\'opez}
\address[Hiram H. L\'opez]{Department of Mathematics and Statistics\\ Cleveland State University\\ Cleveland, OH USA}
\email{h.lopezvaldez@csuohio.edu}
\thanks{}
%%%%%%%%%%%%%%%%%%%%%%%
\author{Nathan Nichols}
\address[Nathan Nichols]{School of Mathematics \\ University of Minnesota Twin Cities \\ Minneapolis, MN USA}
\email{nathannichols454@gmail.com}
%%%%%%%%%%%%%%%%%%%%%%%
\author{Matthew Perkins}
\address[Matthew Perkins]{Department of Mathematics\\ Cleveland State University\\ Cleveland, OH USA}
\email{m.r.perkins73@vikes.csuohio.edu}
%%%%%%%%%%%%%%%%%%%%%%%
\author{Ivan Soprunov}
\address[Ivan Soprunov]{Department of Mathematics\\ Cleveland State University\\ Cleveland, OH USA}
\email{i.soprunov@csuohio.edu}
%%%%%%%%%%%%%%%%%%%%%%%
\author{German Vera-Mart\'inez}
\address[German Vera-Mart\'inez]{Escuela Superior de F\'isica y Matem\'aticas \\ Instituto Polit\'ecnico Nacional\\ Mexico City, Mexico}
\email{gveram1100@alumno.ipn.mx}
%%%%%%%%%%%%%%%%%%%%%%%
\author{Gwyn Whieldon}
\address[Gwyn Whieldon]{}
\email{gwyn.whieldon@gmail.com}
%%%%%%%%%%%%%%%%%%%%%%%

\keywords{Linear codes, locally recoverable codes, Cartesian codes, evaluation codes, Hamming codes}
\subjclass[2010]{Primary 94B05; Secondary 13P25, 14G50, 11T71}


\begin{abstract}
In this {\it Macaulay2} \cite{M2} package we implement an object called {\tt LinearCode}. We implement functions that compute basic parameters and objects associated with a linear code, such as generator and parity check matrices, the dual code, length, dimension, and minimum distance, among others. We implement an object called {\tt EvaluationCode}, a construction which allows to study linear codes using tools of
algebraic geometry and commutative algebra. We implement functions to generate important families of linear codes such as Hamming codes, cyclic codes, Reed--Solomon codes, Reed--Muller codes, Cartesian codes, monomial--Cartesian codes, and toric codes. In addition, we implement functions for the syndrome decoding algorithm and locally recoverable code construction, which are important tools in applications of linear codes. The package \textit{CodingTheory} is available at \url{https://github.com/Macaulay2/Workshop-2020-Cleveland/tree/CodingTheory/CodingTheory}
\end{abstract}

\maketitle
\markleft{Ball, Camps, Chimal-Dzul, Jaramillo-Velez, L\'opez, Nichols, Perkins, Soprunov, Vera, Whieldon}
%%%%%%%%%%%%%%%%%%%%%%%%%%%%%%%%%%%%%%%%%%%%%%%%%%%%%%%%%%%%%%%%%%%%%%%%%%%%%%%%%%%%%%%%%%%%%%%%%%%%%%%%%%%%%%%%%%%%%%%%%%%%%%%%%%%%%%%%%%%%%%%%%%%%%%%%%%%%%%%%%%%%%%%%%%%%%%
\section{Introduction}
Coding theory has been extensively studied since 1949, when Claude Shannon proved in his seminal paper \cite{Shannon} that linear codes can be used to reliably transmit information from a single source to a single receiver through a noisy channel. Since then, coding theory has found many important engineering applications. For example, coding theory has been used in designing reliable data storage systems, radio communication protocols, and in the emerging field of quantum computers. Coding theory has close ties with many areas in mathematics including linear algebra, commutative algebra, algebraic geometry, and combinatorics. 

In this note we introduce the new {\it Macaulay2} \cite{M2} package called {\tt CodingTheory}. The goal of this package is to provide a range of functions for constructing  linear and evaluation codes over finite fields, and for computing some of their main properties. To this aim, we implement two types of objects, {\tt LinearCode} and {\tt EvaluationCode}. The package also includes implementation of functions for generating important families of linear codes like Hamming codes, cyclic codes, Reed-Solomon codes, Reed-Muller codes, Cartesian codes, monomial-Cartesian codes and toric codes. It also has functions for the syndrome decoding algorithm and locally recoverable codes. 

The organization of this note is as follows. In Section~\ref{def} we describe various ways to construct a linear code over a finite field using the \textit{CodingTheory} package. In Section~\ref{pro} we show how to compute the main parameters of a linear code: length, dimension, and minimum distance. We also illustrate how to compute some of the main algebraic objects associated with linear codes like generator and parity check matrices, dual codes, etc. In Section~\ref{eva} we give a brief introduction to evaluation codes and describe some functions implemented to study these objects. In Section~\ref{families} we explain how to create some of the most studied families of linear codes, including Hamming codes, cyclic codes, Reed-Solomon codes, and Reed-Muller codes. Finally, we give instructions on how to create locally recoverable codes.

In this paper we do not attempt to fully explain every function distributed in this package. For a detailed explanation of all functions in the package, we refer to the {\it Macaulay2} help page which can be accessed by
running

\medskip

\begin{verbatim}
i1: viewHelp CodingTheory
\end{verbatim}

\medskip

More information about basics of coding theory can be found in \cite{huf-pless,MacWilliams-Sloane,van-lint}. Constructions of codes using commutative algebra as evaluation codes can be seen in
\cite{carvalho4,GLS,Ha1,LSc,MPV,MPV2,algcodes,renteria-tapia-ca2, Ru, SoSo, So1}. Excellent references for theory of vanishing ideals and their properties are \cite{CLO,monalg}.

\section{Constructing linear codes}\label{def}

Let $\mathbb{F}_q$ be a finite field with $q$ elements. Mathematically, a {\it linear code} is defined as a vector subspace $C\subseteq \mathbb{F}_q^n$ and it is often specified by a {\it generator matrix}, which is a $k \times n$ matrix $G$ with entries in $\mathbb{F}_q$ whose $k$ rows form a basis for $C$. In {\it Macaulay2}, a linear code is defined as an $\mathbb{F}_q$-submodule of $\mathbb{F}_q^n$ using the constructor {\tt linearCode}.  This constructor is an instance of the {\tt LinearCode} type. There are various ways to use the command {\tt linearCode}. For example, one can use this command to construct a linear code $C\subseteq \mathbb{F}^n_q$ by specifying a generator matrix $G$ of $C$ (Example \ref{example_2.1}). Alternatively, one can use the command {\tt linearCode} to construct a linear code $C\subseteq \mathbb{F}_q^n$ by indicating the finite field $\mathbb{F}_q$ and a list $L$ of elements of $\mathbb{F}_q^n$ that span $C$ (Example \ref{example_2.2}).  More details and equivalent ways to use the constructor {\tt linearCode} are given next.

\medskip

\begin{minipage}[t]{0.6\textwidth}
 Inputs:
 \begin{itemize}[leftmargin=0.55 cm]
  \item {\tt F = GF(q)}, a finite field with {\tt q} elements
  \item {\tt n, r, p}, positive integers with {\tt p} prime
  \item {\tt G}, a matrix with entries in {\tt GF(q)}
  \item {\tt L}, a list of elements of $\mathtt{GF(q)}^\mathtt{n}$
 \end{itemize}
\end{minipage}
\begin{minipage}[t]{0.35\textwidth}
Usage:
\begin{itemize}[leftmargin=0.55 cm]
\item {\tt linearCode(G)}
\item {\tt linearCode(F,L)}
\item {\tt linearCode(F,n,L)}
\item {\tt linearCode(p,r,n,L)}
\end{itemize}
\end{minipage}

\medskip

In the next examples we construct a simple binary linear code $C\subseteq \mathbb{F}^4_2$ with generator matrix 
$G=
\begin{pmatrix}
 1 & 1 & 0 & 0\\
 0 & 0 & 1 & 1\\
\end{pmatrix}
$ using equivalent versions of the command {\tt linearCode}.


\begin{example}\label{example_2.1}
 $\,$
\begin{verbatim}
i1 : F = GF(2);
i2 : L = {{1,1,0,0},{0,0,1,1}};
i3 : G = matrix apply(L,codeword->apply(codeword,entry->sub(entry,F)));
i4 : C = linearCode(G);
i5 : C.GeneratorMatrix
o5 = | 1 1 0 0 |
     | 0 0 1 1 |
 \end{verbatim}
\end{example}
\vskip -0.5 cm
Note that in Example \ref{example_2.1} was necessary to coerce the entries of each vector in the list {\tt L} into elements of {\tt F = GF(2)}. An equivalent way to do this is to pass the field {\tt GF(q)} to the {\tt matrix} constructor or to use the constructor {\tt linearCode(GF(q),L)}.

\begin{example}\label{example_2.2}
 $\,$
\begin{verbatim}
i1 : L = {{1,1,0,0},{0,0,1,1}};
i2 : C = linearCode(GF(2),L);
i3 : C.GeneratorMatrix
o3 = | 1 1 0 0 |
     | 0 0 1 1 |
\end{verbatim}
\end{example}
The set $\mathbb{F}^{\ast}_q$ of non-zero elements of a finite field $\mathbb{F}_q$ is a multiplicative cyclic group (\cite[Theorem 3.3.1]{huf-pless}). A generator of $\mathbb{F}_q^{\ast}$ is called a \textit{primitive element} of $\mathbb{F}_q$. One way to refer to a primitive element of a finite field is by specifying a symbol using the {\tt Variable} option of the constructor {\tt GF}. In the next example we illustrate how to define a linear code  $C\subseteq \mathbb{F}_{11}^{10}$ with generator matrix
\[
 G=
 \begin{pmatrix}
  1 & a^1 & a^2 & \cdots & a^9\\
  1 & a^2 & a^4 & \cdots & (a^2)^9\\
  1 & a^3 & (a^3)^2 & \cdots & (a^3)^9\\
  1 & a^4 & (a^4)^2 & \cdots & (a^4)^9\\
 \end{pmatrix},
\]
where $a$ is a primitive element of $\mathbb{F}_{11}$. In {\it Macaulay2}, $a=2$.

\begin{example}\label{example_2.4}
 $\,$
\begin{verbatim}
i1 : F = GF(11,Variable => a);
i2 : G = matrix table({1,2,3,4},
                      {1,a,a^2,a^3,a^4,a^5,a^6,a^7,a^8,a^9},(i,j)->j^i);
i3 : C = linearCode(G);
i4 : C.GeneratorMatrix
o4 = | 1 2  4  -3 5  -1 -2 -4 3  -5 |
     | 1 4  5  -2 3  1  4  5  -2 3  |
     | 1 -3 -2 -5 4  -1 3  2  5  -4 |
     | 1 5  3  4  -2 1  5  3  4  -2 |
\end{verbatim}
\end{example}

In $\mathbb{F}_q^n$ there is a standard inner product defined for all $x=(x_1,\ldots,x_n)$ and $y=(y_1,\ldots,y_n)$ in $\mathbb{F}_q^n$ as $x\cdot y=x_1y_1+\cdots+x_ny_n.$ Given a linear code $C\subseteq \mathbb{F}^n_q$, the {\it dual} or {\it orthogonal code} of $C$ with respect to this product is the linear code $C^{\perp}$ defined as
\[
C^\perp=\{x\in \mathbb{F}_q^n\,:\,x\cdot c=0 \text{ for all }c\in C\}.
\]
A generator matrix of $C^{\perp}$ is called a {\it parity check matrix} for $C$. Since $(C^{\perp})^{\perp}=C$, another common way to mathematically define $C$ is by specifying one of its parity check matrices. By using the command {\tt linearCode} and the {\tt ParityCheck} option of this command, we can define a linear code $C\subseteq \mathbb{F}^n_q$ by either specifying a parity check matrix $H$ of $C$ or a list $L$ of elements of $\mathbb{F}_q^n$ that span $C^{\perp}$.

%We now present some examples to illustrate how to use the {\tt ParityCheck} option of the command {\tt linearCode}. 
In the next example, we define a linear code $C\subseteq \mathbb{F}_9^5$ whose dual code $C^{\perp}$ is generated by the set $\{(1,0,a,0,0),(0,1,a+1,1,0),(1,1,1,a,0)\}\subseteq \mathbb{F}^5_9$, where $a$ is a primitive element of $\mathbb{F}_9$. In {\it Macaulay2}, $a\in\mathbb{F}_9$ satisfies $a^2=a+1$.

\begin{example}\label{example_2.5}
$\,$
\begin{verbatim}
i1 : F = GF(9,Variable => a);
i2 : L = {{1,0,a,0,0},{0,a,a+1,1,0},{1,1,1,a,0}};
i3 : C = linearCode(F,L,ParityCheck => true);
i4 : C.GeneratorMatrix
o4 = | a-1 0 a+1 1 0 |
     | 0   0 0   0 1 |
i5 : C.ParityCheckMatrix
o5 = | 1 0 a   0 0 |
     | 0 a a+1 1 0 |
     | 1 1 1   a 0 |
\end{verbatim}
\end{example}


Although the dual code of a linear code can be constructed using the command {\tt dualCode} implemented in the {\it CodingTheory} package, the {\tt ParityCheck} option of the command {\tt linearCode} also allow us to construct the dual of a linear code. In the next example we construct the dual of the linear code in Example \ref{example_2.4}.

\begin{example}\label{example_2.5}
 $\,$
 \begin{verbatim}
i1 : F = GF(11,Variable => a);
i2 : G = matrix table({1,2,3,4},
                      {1,a,a^2,a^3,a^4,a^5,a^6,a^7,a^8,a^9},(i,j)->j^i);
i3 : D = linearCode(G,ParityCheck => true);
i4 : D.GeneratorMatrix
o4 = | 1  -3 5  3  1 0 0 0 0 0 |
     | -3 -1 4  -4 0 1 0 0 0 0 |
     | 4  -4 -3 5  0 0 1 0 0 0 |
     | -5 -3 4  4  0 0 0 1 0 0 |
     | -4 -4 -1 3  0 0 0 0 1 0 |
     | -3 5  3  1  0 0 0 0 0 1 |
 \end{verbatim}
\end{example}

\section{Basic parameters of linear codes}~\label{pro}

The {\it dimension\/}, {\it length\/}, and {\it rate} or {\it information rate}  of a linear code are basic parameters. Given a linear code $C\subseteq \mathbb{F}^n_q$, these parameters are defined as $k=\dim_{\mathbb{F}_q} C$, $n$, and $k/n$ respectively. Another parameter of a linear code $C\subseteq \mathbb{F}^n_q$ is the {\it minimum Hamming weight}, which is given by 
\[
w_H(C)=\min\{\|c\|
\colon  c \in C, c \neq 0\},
\]
where $\| c \|$ is the number of non-zero entries of $c\in \mathbb{F}^n_q$. This parameter is important in determining the error-correcting capability of $C$; the higher the minimum Hamming weight, the more errors the code can detect and correct (see \cite{huf-pless}). This function has been implemented \textcolor{red}{Nathan, write here something about the implementation of the weight function}.

In {\it Macaulay2}, the space $\mathbb{F}_q^n$ has been called the {\it ambient module} or {\it ambient space} of $C$, while the field $\mathbb{F}_q$, the {\it alphabet} or {\it field} of $C$. As often, the elements of $C$ are referred to as {\it codewords}. The input in all the following commands implemented in the {\tt CodingTheory} package is always a linear code  {\tt C}.
 
\medskip
  
\begin{minipage}[t]{0.3\textwidth}
\begin{itemize}[leftmargin=0.5 cm]
\item {\tt dim C}
\item {\tt length C}
\item {\tt alphabet C}
\item {\tt ambientSpace C}
\end{itemize}
\end{minipage}
\begin{minipage}[t]{0.32\textwidth}
\begin{itemize}[leftmargin=0.35 cm]
\item {\tt field C}
\item {\tt codewords C}
\item {\tt C.Generators}
\item {\tt C.GeneratorMatrix}
\end{itemize}
\end{minipage}
\begin{minipage}[t]{0.35\textwidth}
\begin{itemize}[leftmargin=0.55 cm]
\item {\tt C.AmbientModule}
\item {\tt minimumWeight C}
\item {\tt informationRate C}
\item {\tt C.ParityCheckMatrix}
\end{itemize}
\end{minipage}
\newpage
\begin{example}
$\,$
\begin{verbatim}
i1 : L = {{1,1,0,0},{0,0,1,1}};
i2 : C = linearCode(GF(4),L);
i3 : dim C
o3 : 2
i4 : length C
o4 = 4
i5 : alphabet C
o5 = {0, a, a + 1, 1}
i6 : ambientSpace C
           4
o6 = (GF 4)
i7 : field C
o7 = GF 4
i8 : minimumWeight C
o8 = 2
i9 : codewords C
o9 = {{1, 1, 1, 1}, {1, 1, a, a}, {a, a, 1, 1}, {a, a, a, a}, 
     ------------------------------------------------------------------
      {a + 1, a + 1, a, a}, {a + 1, a + 1, 1, 1}, {1, 1, a + 1, a + 1},
     ------------------------------------------------------------------
      {a, a, a + 1, a + 1}, {a + 1, a + 1, a + 1, a + 1}, {1, 1, 0, 0},
     ------------------------------------------------------------------
      {0, 0, 1, 1}, {0, 0, a, a}, {a, a, 0, 0}, {a + 1, a + 1, 0, 0}, 
     ------------------------------------------------------------------
      {0, 0, a + 1, a + 1}, {0, 0, 0, 0}}
\end{verbatim}
\end{example}
\section{Evaluation codes}\label{eva}
Let $P=\left\{{\bf a}_1,\ldots,{\bf a}_n\right\}$ be a subset of $\mathbb{F}_q^{m}$.
Consider a finite dimensional subspace $\mathcal{S}\subset {\mathbb{F}_q}[X_1,\ldots,X_m]$  of the ring of polynomials over $\mathbb{F}_q$ in $m$ variables. The {\it evaluation map\/}
\begin{equation*}\label{functionevaluation}
{\rm ev}_\mathcal{S}\colon \mathcal{S}\longrightarrow {\mathbb{F}_q}^{|P|},\ \ \ \ \ 
f\mapsto \left(f({\bf a}_1),\ldots,f({\bf a}_n)\right),
\end{equation*}
defines a linear map of ${\mathbb{F}_q}$-vector spaces. The image of ${\rm ev}_\mathcal{S}$ in ${\mathbb{F}_q}^{|P|}$, denoted by $C_{P}(\mathcal{S})$, %defines a ${\mathbb{F}_q}$-vector subspace. We refer to $C_{P}(d)$ 
is the {\it evaluation code\/} on the set $P$ corresponding to $\mathcal{S}$. The {\it vanishing ideal} of $P$, denoted by $I(P)$, is the ideal in $\mathbb{F}_q[X_1,\ldots,X_n]$ of all polynomials that vanish on  $P$. A key observation that allows the use of commutative algebra in studying evaluation codes is that the kernel of the evaluation map ${\rm ev}_S$ is precisely $\mathcal{S}\cap I(P)$.

An evaluation code $C_{P}(\mathcal{S})$ is implemented in {\it Macaulay2} as an {\tt EvaluationCode}. Although, the object {\tt C.LinearCode} is a linear code in {\it Macaulay2}. This has been done in this way because there are more objects associated with an evaluation code than with a linear code. For instance, the vanishing ideal associated to the set $P$ plays an important role when finding and estimating parameters of the code, so it is convenient to be able to access it.

There are many constructions of evaluation codes for specific choices of the set $P$ and the subspace $\mathcal{S}$. These include
Reed-Muller codes, Cartesian, monomial Cartesian codes, toric codes, and evaluation codes from graphs. We refer to 
\cite{carvalho4, Ha1, LSc, lopez-villa, MPV, algcodes, Ru, SoSo} for details on how these codes are defined and what properties they have from coding theory, commutative algebra, and algebraic geometry perspectives. 



Some functions implemented in the {\it CodingTheory} package for various constructions of evaluation codes and associated algebraic objects are the following:

\medskip

Inputs:

\medskip

\begin{minipage}[t]{0.53\textwidth}
\begin{itemize}[leftmargin=0.55 cm]
  \item {\tt I}, an ideal
  \item {\tt d,r}, positive integers
  \item {\tt F = GF(q)}, a finite field with {\tt q} elements
  \item {\tt P}, a list of points in $\mathtt{F}^\mathtt{m}$
  \item {\tt S}, a list of polynomials in {\tt m} variables
\end{itemize}
\end{minipage}
\begin{minipage}[t]{0.45\textwidth}
\begin{itemize}[leftmargin=0.55 cm]
  \item {\tt M}, an integer matrix
  \item {\tt L}, a list of subsets of {\tt F}
  \item {\tt v}, a list of {\tt m} positive integers
  \item {\tt MI}, an incident matrix of a graph
 \end{itemize}
\end{minipage}


\medskip

Usage:

\medskip


\begin{minipage}[t]{0.53\textwidth}
\begin{itemize}[leftmargin=0.55 cm]
\item {\tt evaluationCode(F,P,S)}
\item {\tt toricCode(F,M)}
\item {\tt cartesianCode(F,L,d)}
\item {\tt orderCode(F,P,v,d)}
\item {\tt evCodeGraph(F,MI,S)}
\end{itemize}
\end{minipage}
\begin{minipage}[t]{0.45\textwidth}
\begin{itemize}[leftmargin=0.55 cm]
\item {\tt vNumber(I)}
\item {\tt footPrint(d,r,I)}
\item {\tt hYp(d,r,I)}
\item {\tt genMinDisIdeal(d,r,I)}
\item {\tt vasconcelosDegree(d,r,I)}
\end{itemize}
\end{minipage}

\medskip

The mathematical definitions of the functions in the second column of the previous list can be found in \cite{CSTVV}. The following example shows how to construct an evaluation code in {\it Macaulay2} using the {\it CodingTheory} package.

\begin{example}
$\,$
\begin{verbatim}
i1 : F=GF(4,Variable=>a); R=F[x,y,z];
i3 : P={{0,0,0},{1,0,0},{0,1,0},{0,0,1},{1,1,1},{a,a,a}};
i4 : S={x+y+z,a+y*z^2,z^2,x+y+z+z^2};
i5 : C=evaluationCode(F,P,S);
i6 : (C.LinearCode).GeneratorMatrix
o6 = | 0 1 1 1 1   a   |
     | a a a a a+1 a+1 |
     | 0 0 0 1 1   a+1 |
     | 0 1 1 0 0   1   |

i7 : length C.LinearCode
o7 = 6
i8 : dim C.LinearCode
o8 = 3
i9 : C.Points
o9 = {{0, 0, 0}, {1, 0, 0}, {0, 1, 0}, {0, 0, 1}, {1, 1, 1}, {a, a, a}}

i10 : C.VanishingIdeal

                         2    2                      2    2              
o10 = ideal (x*z + y*z, y  + z  + y + z, x*y + y*z, x  + z  + x + z, 
      --------------------------------------------------------------
      3           2
     z  + (a +1)z  + a*z)
\end{verbatim}
\end{example}


\section{Families of linear codes}\label{families}

Various important families of linear codes have been defined through the development of coding theory. These include Hamming codes, cyclic codes, Reed-Solomon codes, Reed-Muller codes, etc. The mathematical definitions of these and many other families of codes can be found in \cite{huf-pless,MacWilliams-Sloane,van-lint}.

The following functions have implemented in the {\it CodingTheory} package to construct some of these important families of codes.

\medskip
 
Inputs:

\medskip

\begin{minipage}[t]{0.53\textwidth}
\begin{itemize}[leftmargin=0.55 cm]
  \item {\tt n, k, d, m, s}, positive integers
  \item {\tt F = GF(q)}, a finite field with {\tt q} elements
  \item {\tt g(x)}, a polynomial in {\tt F[x]}
 \end{itemize}
\end{minipage}
\begin{minipage}[t]{0.45\textwidth}
 \begin{itemize}[leftmargin=0.55 cm]
  \item {\tt E}, a list of elements of {\tt F}
  \item {\tt L}, a list of vectors in $F^{\mathtt{m}}$
 \end{itemize}
\end{minipage}

\medskip

Usage:

\medskip

\begin{minipage}[t]{0.35\textwidth}
\begin{itemize}[leftmargin=0.55 cm]
\item {\tt hammingCode(q,s)}
\item {\tt reedSolomonCode(F,E,s)}
\item {\tt reedMullerCode(q,m,d)}
\end{itemize}
\end{minipage}
\begin{minipage}[t]{0.33\textwidth}
\begin{itemize}[leftmargin=0.55 cm]
\item {\tt cyclicCode(F,g(x),n)}
\item {\tt repetitionCode(F,n)}
\item {\tt randomCode(F,n,k)}
%\item {\tt quasiCyclicCode(F,L)}
\end{itemize}
\end{minipage}
\begin{minipage}[t]{0.3\textwidth}
\begin{itemize}[leftmargin=0.55 cm]
\item {\tt zeroCode(F,n)}
\item {\tt universeCode(F,n)}
\item {\tt zeroSumCode(F,n)}
\end{itemize}
\end{minipage}

\medskip

\begin{example}
$\,$
\begin{verbatim}
i1 : C = hammingCode(2,3);
i2 : C.GeneratorMatrix
o2 = | 1 1 1 1 0 0 0 |
     | 1 1 0 0 1 0 0 |
     | 0 1 1 0 0 1 0 |
     | 1 0 1 0 0 0 1 |

i3 : F = GF(5); R = F[x]; g = x-1; C = cyclicCode(F,g,6); 
i7 : C.GeneratorMatrix
o7 = | -1 1  0  0  0  0 |
     | 0  -1 1  0  0  0 |
     | 0  0  -1 1  0  0 |
     | 0  0  0  -1 1  0 |
     | 0  0  0  0  -1 1 |

i8 : C = reedSolomonCode(GF(5),{1,2,3},3);
i9 : (C.LinearCode).GeneratorMatrix
o9 = | 1 1  1  |
     | 1 2  -2 |
     | 1 -1 -1 |
\end{verbatim}
\end{example}
\section{Applications of linear codes}
An important aspect in coding theory is {\it decoding}, which is used when information is transmitted trough a noisy channel. In a few words the idea is the following. Take a vector $c\in C.$ Change the value of some of the entries of $c$ to obtain a new vector $v$. Decoding the vector $v$ means to recover the vector $c$ when only $v$ and $C$ are given. Detailed treatment of decoding algorithms can be found in \cite{huf-pless}. Another, more recent application of coding theory is found in distributed and cloud storage systems. The idea is to use {\it locally recoverable codes}, which are linear codes with the property that every entry can be recovered from a few other entries. For more information on locally recoverable codes see \cite{TamoBarg}. 


Some of the most important functions implemented in the {\it CodingTheory} package that can be used for applications of coding theory are the following:

\medskip

Inputs:

\medskip

\begin{itemize}
 \item {\tt C}, a linear code over {\tt GF(q)}
 \item {\tt v}, a vector in the ambient space of {\tt C}
 \item {\tt \{q,n,k,r\}}, where {\tt q} is a prime power, and {\tt n}, {\tt k} and
      {\tt r} are positive integers
 \item {\tt L}, a pairwise disjoint list of subsets of elements of {\tt GF(q)}
\end{itemize}

\medskip

Usage

\medskip

\begin{itemize}
\item {\tt syndromeDecode(C,$v$,minimumWeight(C))}
\item {\tt LocallyRecoverableCode(\{q,n,k,r\},L,a polynomial)}
\end{itemize}

\medskip


\begin{example}
$\,$
\begin{verbatim}
i1 : C = hammingCode(2,3);
i2 : msg = matrix {{1,0,1,0}};
i3 : v = msg*(C.GeneratorMatrix)
o3 = | 0 1 0 1 0 1 0 |
i4 : err = matrix take(random entries basis source v, 1)
o4 = | 0 0 0 0 1 0 0 |
i5 : received = transpose(transpose (v+err))
o5 = | 0 1 0 1 1 1 0 |
i6 : transpose syndromeDecode(C, transpose recieved, 3)
o6 = | 0 1 0 1 0 1 0 |
\end{verbatim}
\end{example}


\section*{Acknowledgments}
We thank Federico Galetto, Courtney Gibbons, Hiram L\'opez, and Branden Stone for organizing the Macualay2 workshop at Cleveland State University, where this collaboration started. We want to give a special thanks to Branden Stone for helping us to develop the package during the workshop. This work was partially supported by the NSF grant DMS-2003883.

\begin{thebibliography}{99}

\bibitem{carvalho4} C. Carvalho, V. G. Neumann and H. H. L\'opez,
Projective Nested Cartesian Codes,
Bull Braz Math Soc, New Series (2016).

\bibitem{CSTVV} S. M. Cooper, A. Seceleanu, S. O. Toh\v{a}neanu,
M. Vaz Pinto and R. H. Villarreal, 
Generalized minimum distance functions and algebraic invariants of
Geramita ideals, Adv. in Appl. Math. {\bf 112} (2020), 101940.

\bibitem{CLO}D. Cox, J. Little and D. O-Shea,
{\it Ideals, Varieties, and Algorithms}, Springer-Verlag, 1992.

\bibitem{GLS} L. Gold, J. Little, H. Schenck,
{Cayley-Bacharach and evaluation codes on complete intersections},
 J. Pure Appl. Algebra  {\bf 196} (1) (2005) 91--99.

\bibitem{M2} D. R. Grayson and M. E. Stillman,
{\it Macaulay2}, a software system for research in algebraic geometry.
Available at \url{http://www.math.uiuc.edu/Macaulay2/}.

\bibitem{Ha1} J. Hansen, { Toric Surfaces and Error--correcting
    Codes} in Coding Theory, Cryptography, and Related Areas, Springer
 (2000), pp. 132-142.

\bibitem{huf-pless} W. Huffman and V. Pless,
{\it Fundamentals of Error-Correcting Codes},
Cambridge University Press, Cambridge, 2003.

\bibitem{LSc} J. Little, H. Schenck,
{\em Toric surface codes and Minkowski sums},
SIAM J. Discrete Math. {\bf  20}  (2006),  no. 4, 999--1014 (electronic).

\bibitem{lopez-villa} H. H. L\'opez, C. Renter\'ia-M\'arquez and R. H. Villarreal,
Affine Cartesian codes,
Des.\ Codes Cryptogr. \textbf{71}(1) (2014) 5--19.

\bibitem{MacWilliams-Sloane} F. J. MacWilliams and N. J. A. Sloane, 
{\it The Theory of Error-correcting Codes}, North-Holland, 1977.

\bibitem{MPV} J. Mart\'inez-Bernal, Y. Pitones and R. H. Villarreal,
Minimum distance functions of graded ideals and Reed-Muller-type codes,
J. Pure Appl. Algebra 221 (2017), 251--275.


\bibitem{MPV2} J. Mart\'inez-Bernal, Y. Pitones and R. H. Villarreal,
Minimum distance functions of complete intersections,
J. Algebra Appl. 17 (2018), no. 11, 1850204 (22 pages).

\bibitem{algcodes} C. Renter\'\i a, A. Simis and R. Villarreal,
Algebraic methods for parameterized codes and invariants of vanishing ideals over finite fields,
Finite Fields Appl. {\bf 17} (2011), no. 1, 81--104.

\bibitem{Ru}D. Ruano, 
{On the parameters of $r$-dimensional toric codes},
Finite Fields and Their Applications {\bf 13} (2007), pp. 962--976.

\bibitem{Shannon} C. Shannon,
A Mathematical Theory of Communication,
The Bell System Technical Journal, {\bf 27} (1948), no. 3, 379--423.

\bibitem{SoSo} I.~Soprunov, J.~Soprunova, {Toric surface codes and Minkowski length of polygons}, SIAM J. Discrete Math. {\bf 23}, Issue 1, (2009) 384-400 

\bibitem{So1} I. Soprunov,  {Toric complete intersection codes.} Journal of Symbolic Computation. {\bf 50}, (2013) 374--385.

\bibitem{TamoBarg} I. Tamo and A. Barg,
A Family of Optimal Locally Recoverable Codes,
IEEE Transactions on Information Theory {\bf 60} (2014), no. 8, 4661--4676.

\bibitem{renteria-tapia-ca2} C. Renter\'\i a and H. Tapia-Recillas, 
Reed-Muller codes: an ideal theory approach, Comm. Algebra {\bf 25}
(1997), no. 2, 401--413.

\bibitem{van-lint} J. H. Van Lint,
{ Introduction to coding theory},
Third edition, Graduate Texts in Mathematics {\bf 86}, Springer-Verlag, Berlin, 1999.

\bibitem{monalg} R. H. Villarreal, {\it Monomial Algebras\/},
Second edition, 
Monographs and Research Notes in Mathematics, Chapman and Hall/CRC,
Boca Raton, FL, 2015.
\end{thebibliography}
\end{document}
